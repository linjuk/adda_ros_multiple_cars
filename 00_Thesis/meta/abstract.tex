\begin{abstract}[1]
	
Decision-making task is one of the most determinant bonds for constructing an autonomous system. Making solid decisions by foreseeing and estimating future consequences on its own, it what makes autonomous systems intelligent. Decision making on its own is already complex task, but for vehicles, it makes more complex because of the uncertainty of the real world and continues vehicles' interaction with other vehicles and obstacles. Sensors which are using for real-world understanding and features as speed, position, other objects of traffic, etc. are noisy and very dependables from external conditions. But again, it is very hard to measure others road users' intentions due to its randomness, additionally, completely or partially visible obstacles of the road can make any received measurements and information useless. The unit responsible for decision making has to be sensible for these issues and be able to foresee the future conditions that could develop in an endless number of ways to achieve the final goal with the maximum reward or, in other words, with a minimum cost of the process.

\vspace{0.5em}

{\color{red}TODO: add part about what was done in the thesis (at the very end).}



Removing a driver from behind the wheel takes away more than just the physical responses. It also eliminates the complex decision-making that goes into even routine journeys – choosing whether to swerve into a neighboring lane to avoid a possible obstacle or navigating ambiguous intersections.
\end{abstract}
%
%
\selectlanguage{ngerman} % select german language
\begin{abstract}[2]
	Hier können Sie Ihre deutsche Zusammenfassung schreiben. %
\end{abstract}
\selectlanguage{english} % reset to english language
